\documentclass{llncs}

\usepackage{amsfonts}
\usepackage{amsmath}
\usepackage{color}
\usepackage{listings}
\usepackage{microtype}
\usepackage{prooftree}

\author{Dominic P. Mulligan}
\title{Mosquito}
\subtitle{An implementation of higher-order logic}
\titlerunning{Mosquito}
\institute{Computer Laboratory, University of Cambridge}

\newcommand{\act}{\cdot}
\newcommand{\deffont}[1]{\textbf{#1}}
\newcommand{\ent}{\vdash}
\newcommand{\lam}[1]{\lambda{#1}.}
\newcommand{\rulefont}[1]{\ensuremath{(\textbf{#1})}}
\newcommand{\tf}[1]{\mathsf{#1}}

\definecolor{mygreen}{rgb}{0,0.6,0}
\definecolor{mygray}{rgb}{0.5,0.5,0.5}
\definecolor{mymauve}{rgb}{0.58,0,0.82}

\lstset{ %
  backgroundcolor=\color{white},   % choose the background color; you must add \usepackage{color} or \usepackage{xcolor}
  basicstyle=\scriptsize\tt,        % the size of the fonts that are used for the code
  breakatwhitespace=false,         % sets if automatic breaks should only happen at whitespace
  breaklines=true,                 % sets automatic line breaking
  captionpos=b,                    % sets the caption-position to bottom
  commentstyle=\color{mygreen},    % comment style
  %deletekeywords={...},            % if you want to delete keywords from the given language
  %escapeinside={\%*}{*)},          % if you want to add LaTeX within your code
  extendedchars=true,              % lets you use non-ASCII characters; for 8-bits encodings only, does not work with UTF-8
  %frame=single,                    % adds a frame around the code
  keepspaces=true,                 % keeps spaces in text, useful for keeping indentation of code (possibly needs columns=flexible)
  keywordstyle=\color{blue},       % keyword style
  language=haskell,                 % the language of the code
  %morekeywords={*,...},            % if you want to add more keywords to the set
  %numbers=left,                    % where to put the line-numbers; possible values are (none, left, right)
  %numbersep=5pt,                   % how far the line-numbers are from the code
  %numberstyle=\tiny\color{mygray}, % the style that is used for the line-numbers
  %rulecolor=\color{black},         % if not set, the frame-color may be changed on line-breaks within not-black text (e.g. comments (green here))
  showspaces=false,                % show spaces everywhere adding particular underscores; it overrides 'showstringspaces'
  showstringspaces=false,          % underline spaces within strings only
  showtabs=false,                  % show tabs within strings adding particular underscores
  %stepnumber=2,                    % the step between two line-numbers. If it's 1, each line will be numbered
  stringstyle=\color{mymauve},     % string literal style
  tabsize=2,                       % sets default tabsize to 2 spaces
  title=\lstname                   % show the filename of files included with \lstinputlisting; also try caption instead of title
}

\begin{document}

\maketitle

\begin{abstract}
\end{abstract}

\section{Introduction}
\label{sect.introduction}

\subsection{The problem with state}
\label{subsect.problem.with.state}

\section{Mosquito's logic}
\label{sect.mosquito.logic}

Mosquito implements a polymorphically typed higher-order logic similar in style and substance to the implemented logics of existing `HOLs' such as HOL-4, HOL-Light and Isabelle/HOL.

HOL types consist of type variables (we use Greek lowercase letters to range over type variables) and type operators applied to a list of types of length matching their declared arity.
The Mosquito logic is equipped with two primitive type operators, the type of propositions $\mathtt{Bool}$ of arity zero, and the function space $\phi \rightarrow \psi$ of arity two (we take the liberty of writing the function space arrow infix here and throughout the sequel).
New type operators may be introduced via a single function exposed in the kernel to maintain consistency.

Terms

\begin{definition}
\label{defn.finite.support}
Assume $f$ is an automorphism on $\mathbb{V}$.
Define the set:
\begin{displaymath}
fixed_f = \{ a \in \mathbb{V} \mid f(a) = a \}
\end{displaymath}
Call a function $f$ \deffont{finitely supported} when $fixed_f$ is finite.
\end{definition}

\begin{definition}
\label{defn.swapping.action}
Define \deffont{permutations} as finitely supported bijections on variables.
We use $\pi, \pi', \pi''$ and so on to range over permutations.
Extend this definition to a \deffont{swapping action} on terms by the following rules:
\begin{gather*}
\pi \act a \equiv \pi(a) \quad \pi \act \tf{C} \equiv \tf{C} \quad \pi \act rs \equiv (\pi \act r)(\pi \act s) \quad \pi \act \lam{a{:}\phi}r \equiv \lam{\pi(a){:}\phi}(\pi \act r)
\end{gather*}
\end{definition}

\begin{figure}
\begin{gather*}
\begin{prooftree}
(a, b \not\in fv(t))
\justifies
\{ \} \ent t = (a\ b) \act t
\using\rulefont{swap}
\end{prooftree}
\quad
\begin{prooftree}
\Gamma \ent t = u
\justifies
\Gamma \ent u = t
\using\rulefont{sym}
\end{prooftree}
\quad
\begin{prooftree}
\Gamma \ent t = u \quad \Gamma \ent u = v
\justifies
\Gamma \ent t = v
\using\rulefont{trans}
\end{prooftree}
\\[1.5ex]
\begin{prooftree}
\Gamma \ent f = g \quad \Delta \ent t = u
\justifies
\Gamma \cup \Delta \ent f\ t = g\ u
\using\rulefont{app}
\end{prooftree}
\quad
\begin{prooftree}
\Gamma \ent t = u
\justifies
\Gamma \ent \lam{a{:}\phi}t = \lam{a{:}\phi}u
\using\rulefont{lam}
\end{prooftree}
\\[1.5ex]
\begin{prooftree}
(type(t) : bool)
\justifies
\{ t \} \ent t
\using\rulefont{assm}
\end{prooftree}
\quad
\begin{prooftree}
\Gamma \ent t = u \quad \Delta \ent t
\justifies
\Gamma \cup \Delta \ent u
\using\rulefont{eqmp}
\end{prooftree}
\quad
\begin{prooftree}
\Gamma \ent t \quad \Delta \ent u
\justifies
(\Gamma - u) \cup (\Delta - t) \ent t = u
\using\rulefont{asym}
\end{prooftree}
\\[1.5ex]
\begin{prooftree}
\Gamma \ent t
\justifies
\Gamma[\alpha_i := \phi_i] \ent t[\alpha_i := \phi_i]
\using\rulefont{ty-inst}
\end{prooftree}
\quad
\begin{prooftree}
\Gamma \ent t
\justifies
\Gamma[b_i := u_i] \ent t[b_i := u_i]
\using\rulefont{trm-inst}
\end{prooftree}
\\[1.5ex]
\begin{prooftree}
(a \not\in fv(t))
\justifies
\{ \} \ent \lam{a{:}\phi}(t\ a) = t
\using\rulefont{eta}
\end{prooftree}
\quad
\begin{prooftree}
\phantom{h}
\justifies
\{ \} \ent (\lam{a{:}\phi}t)u = t[a := u]
\using\rulefont{beta}
\end{prooftree}
\end{gather*}
\caption{Derivation rules for Mosquito \textsc{hol}}
\label{fig.derivable.equality}
\end{figure}

\section{Tactics}
\label{sect.tactics}

\section{Conclusions}
\label{sect.conclusions}

\end{document}