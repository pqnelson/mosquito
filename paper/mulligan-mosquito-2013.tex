\documentclass{llncs}

\usepackage{amsfonts}
\usepackage{amsmath}
\usepackage{color}
\usepackage[colorlinks]{hyperref}
\usepackage{listings}
\usepackage{microtype}
\usepackage{prooftree}

\author{Dominic P. Mulligan}
\title{Mosquito}
\subtitle{An implementation of higher-order logic}
\titlerunning{Mosquito}
\institute{Computer Laboratory, University of Cambridge}

\newcommand{\act}{\cdot}
\newcommand{\aeq}{=_\alpha}
\newcommand{\deffont}[1]{\textbf{#1}}
\newcommand{\ent}{\vdash}
\newcommand{\lam}[1]{\lambda{#1}.}
\newcommand{\rulefont}[1]{\ensuremath{(\textbf{#1})}}
\newcommand{\tf}[1]{\mathsf{#1}}

\bibliographystyle{alpha}

\definecolor{mygreen}{rgb}{0,0.6,0}
\definecolor{mygray}{rgb}{0.5,0.5,0.5}
\definecolor{mymauve}{rgb}{0.58,0,0.82}

\lstset{ %
  backgroundcolor=\color{white},   % choose the background color; you must add \usepackage{color} or \usepackage{xcolor}
  basicstyle=\scriptsize\tt,        % the size of the fonts that are used for the code
  breakatwhitespace=false,         % sets if automatic breaks should only happen at whitespace
  breaklines=true,                 % sets automatic line breaking
  captionpos=b,                    % sets the caption-position to bottom
  commentstyle=\color{mygreen},    % comment style
  %deletekeywords={...},            % if you want to delete keywords from the given language
  %escapeinside={\%*}{*)},          % if you want to add LaTeX within your code
  extendedchars=true,              % lets you use non-ASCII characters; for 8-bits encodings only, does not work with UTF-8
  %frame=single,                    % adds a frame around the code
  keepspaces=true,                 % keeps spaces in text, useful for keeping indentation of code (possibly needs columns=flexible)
  keywordstyle=\color{blue},       % keyword style
  language=haskell,                 % the language of the code
  %morekeywords={*,...},            % if you want to add more keywords to the set
  %numbers=left,                    % where to put the line-numbers; possible values are (none, left, right)
  %numbersep=5pt,                   % how far the line-numbers are from the code
  %numberstyle=\tiny\color{mygray}, % the style that is used for the line-numbers
  %rulecolor=\color{black},         % if not set, the frame-color may be changed on line-breaks within not-black text (e.g. comments (green here))
  showspaces=false,                % show spaces everywhere adding particular underscores; it overrides 'showstringspaces'
  showstringspaces=false,          % underline spaces within strings only
  showtabs=false,                  % show tabs within strings adding particular underscores
  %stepnumber=2,                    % the step between two line-numbers. If it's 1, each line will be numbered
  stringstyle=\color{mymauve},     % string literal style
  tabsize=2,                       % sets default tabsize to 2 spaces
  title=\lstname                   % show the filename of files included with \lstinputlisting; also try caption instead of title
}

\begin{document}

\maketitle

\begin{abstract}
We present \emph{Mosquito}, a new LCF-style implementation of higher-order logic (`HOL') written in Haskell, a pure functional programming language.
Mosquito requires a kernel design that does not rest on the use of global references, an ML language feature heavily used in existing HOL kernels.
Wiedijk suggested one such design in his work on `Stateless HOL', a stateless revision of the HOL Light kernel.
In one sense, Mosquito is a clean-slate implementation of a Wiedijk-style HOL kernel, able to leverage Haskell's modern language features and rich library ecosystem.

Yet, Mosquito is something more than a mere rehash of existing HOL implementations.
Mosquito is written in an entirely pure, monadic style, shunning any use of exception mechanisms to handle failure, with an emphasis on totality.
The Mosquito kernel is minimalist, arguably implementing a simpler presentation of HOL than HOL Light.
Further, our representation of tactics draws on inspiration from the Matita proof assistant, where working across multiple open goals simultaneously is the norm.
\end{abstract}

\section{Introduction}
\label{sect.introduction}

Mosquito is a minimalist, stateless, monadically-structured, largely-total, LCF-style implementation of higher-order logic (`HOL') written in Haskell.
We shall spend the next ten pages of this paper unpacking, and thereafter expanding, this sentence.

Since the dawn of computer science as an independent subject in its own right computer scientists have attempted to make use of computers to help prove theorems, either purely mathematical in form, or related to the verification of pieces of software.
Due to decidability issues with the varieties of logics most suitable for use in formalisation, so-called `interactive proof assistants', where human ingenuity is paired with brute-force automation and clever heuristics, are popular.

Of critical concern, when developing an interactive proof assistant, is the question of \emph{trust}.
How can users trust the resulting theorems generated by interactive proof assistants?
How can we ensure that our systems remain consistent, never allowing us to prove \texttt{False}?
Two different approaches are common amongst current implementations of proof assistants:
\begin{enumerate}
\item
The 
\item
\end{enumerate}

\subsection{The problem with state}
\label{subsect.problem.with.state}

\section{Kernel}
\label{sect.kernel}

\subsection{Terms and types}
\label{subsect.terms.and.types}

Mosquito's term and type language is what one familiar with existing HOL implementations would expect.

\begin{figure}
\begin{gather*}
\begin{prooftree}
\phantom{h}
\justifies
a \aeq a
\end{prooftree}
\quad
\begin{prooftree}
\phantom{h}
\justifies
\tf{C} \aeq \tf{C}
\end{prooftree}
\quad
\begin{prooftree}
r \aeq r' \quad s \aeq s'
\justifies
rs \aeq r's'
\end{prooftree}
\quad
\begin{prooftree}
r \aeq r'
\justifies
\lam{a{:}\phi}r \aeq \lam{a{:}\phi}r'
\end{prooftree}
\quad
\begin{prooftree}
r \aeq (b\ a) \act s \quad (a \not\in fv(s))
\justifies
\lam{a{:}\phi}r \aeq \lam{b{:}\phi}s
\end{prooftree}
\end{gather*}
\caption{Rules for inductively defining alpha-equivalence}
\label{fig.alpha-equivalence}
\end{figure}

\paragraph{Alpha-equivalence and fresh names}
The native notion of equality on higher-order terms is alpha-equivalence, or equality modulo specific choices of bound variable.
The kernel implements an alpha-equivalence test based on swappings of variables, \emph{a la} Gabbay-Pitts nominal techniques~\cite{gabbay:new:1999}.

Write $(a\ b)$ for the function such that $(a\ b)(a) = b$, $(a\ b)(b) = a$ and $(a\ b)(c) = c$ for all other $c$, assuming a permutative convention where $a$, $b$ and $c$ are assumed to denote distinct variables.
We extend this definition `in the obvious' way to a swapping action on terms $(a\ b) \cdot t$.
This action pushes the swapping through the term structure, renaming $\lambda$-abstracted variables as it passes, until it reaches the leaves of the term, wherein the swapping renames variables and evaporates on constants.
With this definition, define a notion of alpha-equivalence using the rules in Figure~\ref{fig.alpha-equivalence}.


\section{Tactics}
\label{sect.tactics}

\section{Conclusions}
\label{sect.conclusions}

\bibliography{mulligan-mosquito-2013}

\end{document}